\renewcommand\thesection{\arabic{section}}
\renewcommand\thesubsection{\thesection.\arabic{subsection}}
\setcounter{section}{7}
\setcounter{subsection}{0}
\section{Consideraciones generales}

\subsection{Desarrollo de la aplicación}
\label{subsec:desarrolloapp}

Para el desarrollo de esta aplicación, se utilizaron los IDE de desarrollo:
\begin{itemize}
    \item \textbf{IntelliJ Idea:} IDE de desarrollo general de la aplicación.
    \item \textbf{Apache Netbeans:} IDE de desarrollo para diseño de ventanas gráficas (Swing).
\end{itemize}

La versión del JDK de java utilizado es la 1.8 (\faJava \textbf{Java 8}), y también se utilizó \faDocker \textbf{Docker} para la generación de una instancia para correr la base de datos \faDatabase MySQL, junto con PhpMyAdmin para la administración de la misma.

\subsubsection{Codificación y documentación}
\label{subsubsec:codydoc}
En la codificación de la solución se siguieron las buenas prácticas de programación, modularización, principios de OO y encapsulamiento. Además, a medida que se generaban nuevas lineas de código, se fue documentando de forma inmediata y generados los \textbf{Javadocs} correspondientes para cada una de las entregas. Aprovechando el hosting gratuito otorgado por \faGithub \textbf{Github Pages}, los Javadocs generados son accesibles a través {\color{MyGreen}\href{https://sebaignacioo.github.io/proyecto-poo/}{de este link.}}

\subsubsection{Paquetes de Maven}
\label{subsubsec:paqmaven}
En el desarrollo del proyecto se utilizó \textbf{Maven}, lo que permite incorporar librerías de terceros de forma rápida. Las librerías utilizadas para el desarrollo de esta aplicación son las siguientes:

\begin{itemize}
    \item \textbf{Faker:} Librería que permite la generación de datos de prueba falsos.
    \item \textbf{MySQLConnector/J:} Librería oficial de MySQL que permite la conexión con la base de datos.
    \item \textbf{dotenv-java:} Librería para utilizar variables de entorno, para colocar las credenciales de conexión para la base de datos.
    \item \textbf{jIconFont:} Librería que permite obtener íconos predefinidos, para utilizar en el diseño de la interfaz gráfica.
    \item \textbf{FlatLaf:} Librería que permite cambiar los estilos visuales (\textbf{LookAndFeel}) predefinidos para Java Swing.
\end{itemize}

\subsection{Utilización de GitHub}
\label{cons:utilizaciongithub}
Se utilizó \faGithub \textbf{Github} como sistema de control de versiones, lo que facilitó el trabajo en equipo a la hora de codificar la solución. Considerando la entrega final del proyecto, se realizaron 65 commits, repartidos entre las diferentes entregas del proyecto, y subidos por los 3 integrantes del grupo del proyecto.

\subsection{Herramientas de utilidad}
\label{subsec:herramientasutilidad}
Para facilitar el trabajo en equipo, para este proyecto se utilizaron las siguientes herramientas tecnológicas:

\begin{itemize}
    \item \faGithub \textbf{GitHub:} Herramienta para control de versiones. El uso de ramas, en conjunto con el IDE IntelliJ Idea, permitió trabajar en conjunto con el código.
    \item \faTrello \textbf{Trello:} Herramienta para gestión de tiempo y plazos de entrega.
    \item \faDiscord \textbf{Discord:} Herramienta de comunicación para reuniones y juntas de avance.
    \item \faGoogle \textbf{Suite de Google:} Herramientas online para edición de documentos de forma colaborativa.
\end{itemize}
